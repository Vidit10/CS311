\documentclass{article}
\usepackage[margin=0.75in]{geometry}
\usepackage{graphicx}
\usepackage{amsmath}
\usepackage{booktabs}
\usepackage{listings}
\usepackage[colorlinks=true,linkcolor=blue,urlcolor=blue]{hyperref}
\usepackage{multirow}
\usepackage{multicol}
\usepackage{listings}
\usepackage{xcolor}

\definecolor{codegray}{gray}{0.95}

\lstset{
    language=Python,
    backgroundcolor=\color{codegray},
    basicstyle=\ttfamily\footnotesize,
    keywordstyle=\color{blue}\bfseries,
    commentstyle=\color{green!60!black},
    stringstyle=\color{orange},
    numbers=left,
    numberstyle=\tiny\color{gray},
    stepnumber=1,
    numbersep=10pt,
    showspaces=false,
    showstringspaces=false,
    frame=single,
    captionpos=b
}

\title{\textbf{CS311} (Computer Architecture Lab) \textbf{Assignment-06}}
\author{\textbf{Vidit Parikh} [MC23BT010]}
\date{\today}

\pagenumbering{gobble}

\begin{document}

\maketitle

\section{Introduction}
The goal of this assignment was to simulate the addition of caches for a better memory access in the stages which require interaction with the memory. This is done by implementing the following changes:
\begin{itemize}
    \item One cache (\texttt{instCache}) was added between the Instruction Fetch (IF) stage and the main memory. This cache is known as: \texttt{L1i-cache} or the "Level-1 Instruction Cache".
    \item One more kind of cache (\texttt{dataCache}) was added between the Memory Access (MA) stage and the main memory. This cache is known as: \texttt{L1d-cache} or the "Level 1 data cache".
\end{itemize}

\section{Cache Configurations}
{\tiny This is in line with the specifications as mentioned in the assignment problem statement.}
\begin{table}[h!]
    \centering
    \begin{tabular}{|l|l|l|l|l|}
        \hline
        \textbf{Cache Size} & \textbf{16B} & \textbf{128B} & \textbf{512B} & \textbf{1kB}\\
        \hline
        \textbf{Latency} & 1 cycle & 2 cycles & 3 cycles & 4 cycles\\
        \hline
        \textbf{Line Size} & \multicolumn{4}{c|}{4B}\\
        \hline
        \textbf{Associativity} & \multicolumn{4}{c|}{2}\\
        \hline
        \textbf{Write Policy} & \multicolumn{4}{c|}{Write Through}\\
        \hline
    \end{tabular}
    \caption{Cache Configurations}
    \label{tab:cache_config}
\end{table}

\section{Observations \& Tabulations}
Now, in this section, we shall consider two cases --- one, when the size of the data cache is fixed at $1kB (1024B)$, while we vary the instruction cache values to note various observations upon running the simulator program.

\subsection{Varying Instruction Cache}
We fix the size of the \texttt{L1d-cache} at 1kB. Then, we vary the size of the L1i-cache from 16B $\rightarrow$ 128B $\rightarrow$ 512B $\rightarrow$ 1024B (1kB). The latency is supposed to be changed accordingly. Now, we study the performance (IPC - instructions per cycle) and plot the graph as attached.
\begin{table}[h!]
    \centering
    \begin{tabular}{|c|c|c|c|c|c|}
        \hline
        \multirow{2}{*}{\textbf{Name of Program}} & \multicolumn{5}{c|}{\textbf{IPC}} \\
        \cline{2-6}
        & \textbf{No Cache} & \textbf{16B} & \textbf{128B} & \textbf{512B} & \textbf{1024B} \\
        \hline
        \texttt{descending.asm} & 0.02493 & 0.02293 & 0.08848 & 0.07804 & 0.06977 \\
        \hline
        \texttt{evenorodd.asm}  & 0.02439 & 0.02238 & 0.02158 & 0.02083 & 0.02013 \\
        \hline
        \texttt{fibonacci.asm}  & 0.02482 & 0.02291 & 0.05864 & 0.05313 & 0.04850 \\
        \hline
        \texttt{prime.asm}      & 0.02463 & 0.02375 & 0.05022 & 0.04576 & 0.04202 \\  % Changed 0.4202 → 0.04202
        \hline
        \texttt{palindrome.asm} & 0.02462 & 0.02315 & 0.06947 & 0.06113 & 0.05458 \\
        \hline
    \end{tabular}
    \caption{Instruction Per Cycle (IPC) across different L1i-cache sizes}
    \label{tab:ipc_results}
\end{table}

\begin{figure}[h]
    \centering
    \includegraphics[width=0.75\linewidth]{case1.png}
    \caption{Graph for comparing the varying L1i-cache}
    \label{fig:graph1}
\end{figure}

\subsection{Varying Data Cache}
We now fix the size of the \texttt{L1i-cache} at 1kB. Then, we vary the size of the L1d-cache from 16B $\rightarrow$ 128B $\rightarrow$ 512B $\rightarrow$ 1024B (1kB). The latency is supposed to be changed accordingly. Now, we study the performance (IPC - instructions per cycle) and plot the graph as attached.
\begin{table}[h!]
    \centering
    \begin{tabular}{|c|c|c|c|c|c|}
        \hline
        \multirow{2}{*}{\textbf{Name of Program}} & \multicolumn{5}{c|}{\textbf{IPC}} \\
        \cline{2-6}
        & \textbf{No Cache} & \textbf{16B} & \textbf{128B} & \textbf{512B} & \textbf{1024B} \\
        \hline
        \texttt{descending.asm} & 0.02493 & 0.06490 & 0.07075 & 0.07025 & 0.06977 \\
        \hline
        \texttt{evenorodd.asm}  & 0.02439 & 0.02013 & 0.02013 & 0.02013 & 0.02013 \\
        \hline
        \texttt{fibonacci.asm}  & 0.02482 & 0.04903 & 0.04885 & 0.04867 & 0.04850 \\
        \hline
        \texttt{prime.asm}      & 0.02463 & 0.04202 & 0.04202 & 0.04202 & 0.04202 \\  % Changed 0.4202 → 0.04202
        \hline
        \texttt{palindrome.asm} & 0.02462 & 0.05458 & 0.05458 & 0.05458 & 0.05458 \\
        \hline
    \end{tabular}
    \caption{Instruction Per Cycle (IPC) across different L1d-cache sizes}
    \label{tab:ipc_results}
\end{table}

\begin{figure}[h]
    \centering
    \includegraphics[width=0.75\linewidth]{case2.png}
    \caption{Graph for comparing the varying L1d-cache}
    \label{fig:graph2}
\end{figure}

\section{Observations \& Conclusions}
\begin{itemize}
    \item We successfully modelled the updated latency in the simulator after implementing cache memory. The updates have been made in the InstructionFetch (IF) Stage and during the load/store process, under the MemoryAccess (MA) Stage. 
    \item For the code based implementation, separate classes namely - \texttt{Cache.java} and \texttt{CacheLine.java} were created within \texttt{Processor.memorysystem}.
    \item From the graphical representation of the table, we observe that the introduction of an \texttt{L1i-cache} decreases IPC initially, followed by a sharp increase (exception being in case of \texttt{evenoradd.asm}). Additionally, IPC steadily decreases with increase in \texttt{L1i-cache} size.
    \item Also, it can be observed that introduction of an \texttt{L1d-cache} sharply increases IPC initially, followed by converging the IPC on a further increase in the \texttt{L1d-cache size}.
    \item We can conclude that with a considerable amount of data, we can find an optimal value for the sie of \texttt{L1i-cache} and \texttt{L1d-cache} to maximise the throughput (IPC).
\end{itemize}

\newpage

\appendix
\section{Code for Generating Graphs}
\begin{lstlisting}[caption={Code for graph for varying l1i-cache (l1d-constant}]
import matplotlib.pyplot as plt

# Defining the data
X = [0, 16, 128, 512, 1024]
Y_descending = [0.02493, 0.02293, 0.08848, 0.07804, 0.06977]
Y_evenorodd = [0.02439, 0.02238, 0.02158, 0.02083, 0.02013]
Y_fibonacci = [0.02482, 0.02291, 0.05864, 0.05313, 0.04850]
Y_prime = [0.02463, 0.02375, 0.05022, 0.04576, 0.04202]
Y_palindrome = [0.02462, 0.02315, 0.06947, 0.06113, 0.05458]

# Initialise the figure (graph)
plt.figure(figsize=(12, 8))

# Plot each line with label
plt.plot(X, Y_descending, "-o", label="descending.asm")
plt.plot(X, Y_evenorodd, "-s", label="evenorodd.asm")
plt.plot(X, Y_fibonacci, "-^", label="fibonacci.asm")
plt.plot(X, Y_prime, "-d", label="prime.asm")
plt.plot(X, Y_palindrome, "-p", label="palindrome.asm")

# Annotate all points
def annotate_points(X, Y, offset=(0, 10)):
    for x, y in zip(X, Y):
        plt.annotate(f"{y:.5f}", xy=(x, y), textcoords="offset points",
                     xytext=offset, ha='center', fontsize=8)

annotate_points(X, Y_descending, offset=(0, 10))
annotate_points(X, Y_evenorodd, offset=(0, -15))
annotate_points(X, Y_fibonacci, offset=(0, 10))
annotate_points(X, Y_prime, offset=(0, -15))
annotate_points(X, Y_palindrome, offset=(0, 10))

# Axis labels and title
plt.title("Program IPC vs L1i-cache Size (Fixed L1d-cache)", fontsize=16)
plt.xlabel("L1i-cache Size (Bytes)", fontsize=14)
plt.ylabel("Throughput (IPC)", fontsize=14)
plt.xticks(X)
plt.grid(True, linestyle='--', alpha=0.7)
plt.legend(title="Assembly Programs", fontsize=10)
plt.tight_layout()

# Save the figure
plt.savefig("case1.png", dpi=1000)

plt.show()

\end{lstlisting}


\end{document}